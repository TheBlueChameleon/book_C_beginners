
\chapter{Writing C Code}
\epigraph{The secret to getting ahead is getting started.}{Mark Twain}

In this chapter, we'll discuss the principal structure of a C language program by analyzing a so called \emph{Hello World} -- a program which has the sole purpose of putting the text \emph{Hello World} on screen\footnote{The expressin \emph{Hello World} is also used in context of other languages. This \enquote{most simple} possible program is commonly used as a sort of reference, a way to compare programming languages -- or to simply fool around. People have made s sport out of writing programs that essentially are Hello World variations but don't look like anything made for human eyes at all. See, for example, \url{https://codegolf.stackexchange.com/questions/22533/weirdest-obfuscated-hello-world}}.

Once we've familiarized ourselves with this primary structure, we'll start writing actually useful code by making simple calculations in C.

\section{A First Program: Hello World}

The following lines show code that prints the text \texttt{Hello World} on screen:
\begin{codebox}[helloworld.c]
\begin{minted}[linenos]{c}
#include <stdio.h>

int main() {
   printf("Hello World!\n");
}
\end{minted}
\captionof{listing}{A Hello World} \label{code:helloworld}
\end{codebox}

To compile and run it, remember the compiler invocation:

\begin{cmdbox}[Compiling and Executing the Hello World]
\$ gcc -std=c17 -Wall -Wextra -Wpedantic helloWorld.c -o helloWorld \\
\$ ./helloWorld \\
Hello World!
\end{cmdbox}

Now let's go through listing \ref{code:helloworld} line by line.

We've already discussed that recurrent tasks can be externalized and put into a library. Input and Output fram and to the console is such a recurrent task. The routines concerned with these tasks are implemented in the \emph{C standard library}, which is automatically included in the linkage process -- no need to use a compiler option. However, these pre-compiled routines are, well, compiled, \ie translated into machine language. In this process, some information is lost that would allow for direct integration in our code. To provide this lost information, a so called \emph{Header File} needs to be \emph{included} into our code. For the time being, don't concern yourself with what this means in detail too much; simply think of a header file as a sort of glue that helps combining your C-Code to machine language code somewhere else.

After these explanations, you can probably guess that in line 1 we include this header file. The command \texttt{\#include} loads a file from the hard disk, and copy-pastes it into the C-code file. The file to load is specified in <angle brackets>\footnote{In chapter \ref{chp:Preprocessor}, we'll see that there are other options, too -- but let's not get ahead of ourselves.} This also constitutes a preprocessor operation: line 1 is replaced by the content of the file \texttt{stdio.h} by the preprocessor. All \emph{preprocessor directives} begin with a pound symbol (\texttt{\#}).

Programs are organized into subroutines or \emph{functions}, \ie smaller sections of code that form the building blocks of the whole. Line 3 \emph{declares} such a function. It has the name \texttt{main}. The \emph{body} of the function (\ie its \enquote{content}) is enclosed by \{curly braces\}. All code but a very few needs to be placed within functions, so we'll see this pattern in every program we'll see throughout our C coding days. Functions can have virtually any name, but the name \texttt{main} is special: it marks the, well, \emph{main} function, \ie the starting point for our program. When there are multiple functions, all of them but the \texttt{main} function are \enquote{ignored} until you as a programmer explicitly \emph{call} them -- but we'll hear more about that in chapter \ref{chp:funcs}. For now, accept simply that the line \inC{int main ()} marks the starting point for the \enquote{proper} code, which you will find within \{braces\}.

The only line left to understand is line 4. It contains a \emph{statement}, \ie a command that can be executed, and this statement is a \emph{call to the function \texttt{printf}}. That means, somewhere (in the precompiled standard library), there is another function with the name \texttt{printf}. The code there can be used to make output on screen (the \texttt{f} in the function's name stands for \emph{formatted} -- we'll see later, why that is). Of course, the routine needs to know \emph{what text} should be put on screen. To distinguish the \emph{function arguments} (\ie the information sent or \emph{passed} to the function) from the rest of the code, they are enclosed by (parens). In general, we can pass all kinds of information into functions. In the previous chapter, we've already heard that there is some ambiguity as to how to interpret \enquote{raw information}. In particular, we've seen that text is treated differently from numbers. But also when we're dealing with text (like our C code), there can still be different ways to interpret them. \texttt{Hello Word!} could be the text we want to print on screen, or some C code that needs to be translated into machine language. Since the computer can't guess our intent (remember our mantra from the preface?), we need to be explicit: we mark the text we want to put on screen as \emph{literal text} by enclosing it in a pair of \texttt{"}quotation marks\texttt{"}.

While this makes it clear what is code and what is literal text, it also introduces some minor problems. Say, you want to have some text that also includes quotation marks. There needs to be a way to tell the compiler that a double quote in the code is not to be interpreted as end of text delimiter but as part of the text. This is done by \emph{escape sequences}: special substrings in a text that are not taken verbatim but interpreted in a different (simple) way. In C, escape sequences begin with a backslash (\textbackslash) followed by usually one chacracter. For example, the sequence \texttt{\textbackslash"} becomes such a quotation mark within the text.

In line four, we find the escape sequence \texttt{\textbackslash n}; this encodes a line break (\texttt{n} stands for \emph{new line}). Unfortunately, literal texts cannot span multiple lines, so for the time being, these escape sequences are the only way to make multiline output. See table \ref{tab:Escape} in the appendix for more escape sequences.

Note also that line four ends with a semicolon (;). This ends the statement. Each statement needs to end in a semicolon, otherwise, you'll find that gcc refuses to compile your code and outputs an error message.

\begin{cmdbox}[Error message when compiling \texttt{helloworld.c} without semicolon in line 4]
\begin{minted}{text}
helloworld.c: In function ‘main’:
helloworld.c:4:29: error: expected ‘;’ before ‘}’ token
    4 |     printf("Hello World!\n")
      |                             ^
      |                             ;
    5 | }
      | ~
\end{minted}
\end{cmdbox}

This is because (unlike literal strings) statements can be spread over multiple lines. Similarly, there can be more than one statement in a single line. As programmers, this gives us some liberty in formatting the code such that it can be read more easily. Also, we can add or remove whitespaces (or tabs) pretty much anywhere, as long as we keep \enquote{words} together. The following code produces the same output as listing \ref{code:helloworld}
\begin{codebox}[Code with optional whitespaces added]
\begin{minted}[linenos]{c}
  #include     <stdio.h>
 int    main   (   )    {
      printf   (   
         "Hello World!\n"
)   ;
   }
\end{minted}
\captionof{listing}{A Hello World with added whitespaces} \label{code:optionalWhitespaces}
\end{codebox}

\begin{hintbox}[Rule of thumb: whitespaces]
The added whitespaces in listing \ref{code:optionalWhitespaces} are meant to illustrate where it is allowed to add them; clearly, this example makes the code look more like a random image than an organized list of instructions. This is not to say that \emph{indenting} your code is a bad idea. Quite to the contrary, these whitespaces and extra line breaks can help you a lot in visualizing relationships between different parts of the code or filter relevant information. This becomes paramount once our codes reach some complexity.

As a rule of thumb, you should add a level of indentation for every opening brace (\{\}), paren () or bracket ([]) that is not closed in the same line. Furthermore, the second to last line of statements that take more than one line of code should also be indented.

It is up to you, how many whitespaces (or tabs) you add; only you should stay consistent throughout your code. Commonly, one indentation level means four white spaces, but also two or eight whitespaces are not unheard of. Many code editors can be configured to automatically indent to the next multiple of two, four or eight whitespaces on the press of the tabulator key.

In this book I'll do my best to stay consistent in my indentation style. You may take this as inspiration for your own code.
\end{hintbox}

The gcc reads the code \enquote{from top to bottom}. This means that the order of statements is of great importance. If, for example, you move the \inC{#include} statement from line 1 to the end of the code, you'd get an error message. The reason behind this is that the command \texttt{printf} is only declared in the fine \texttt{stdio.h}; before this declaration, the compiler \enquote{doesn't know} what \texttt{printf} even means.

\section{Variables, Data Types, Operators} \label{sec:expressions}
\emph{Useful} programs evolve over time. They compute some result or react to user input. To realize such dynamicity, C allows the existens of \emph{variables}, \ie stored information that can change over time. In this section we'll learn how to create and use variables and some of the \enquote{behind the scenes}, \ie what happens in memory when we use them.

\subsection{Memory Structure}
Imagine the memory as a long chain of enumerated cells. Each cell holds one byte's worth of information, \ie eight bits. These cells can never be empty; there's always \emph{some} information stored in them, albeit possibly nonsensical or useless. If we want to make use of this structure, someone or something will have to keep track of which cells hold which pieces of information, and how to interpret them.

Luckily, most of this bookkeeping is done automatically by the compiler; to properly utilize its features, however, we need to clarify some names.

Within the cells, we find \emph{values}. As mentioned, they are, a priori, pure bitpatterns with no inherent meaning. Provided we know \emph{what sort of information} they encode, we can make sense of these bitpatterns.

The question \emph{what sort of information} is contained in a cell is called \emph{data type}. The data type is a name for both, the method of interpretation of the raw data in the cells, as well as the \emph{number of bytes per unit of information}. From the previous chapter, you remember how to interpret bit patterns as signed integers; to do so, you also need to know how many bytes to consider in your interpretation. A \emph{32bit signed integer} is thus a data type.

The value we are looking for is in one of the enumerated cells. The \enquote{number of the cell} in which a given value is stored is its \emph{address}. An address itself can be stored as some unsigned integer. Adresses often are expressed as \emph{hexadecimal numbers}. That means, the number is written in base 16. You'll not only find the digits 0..9, but also A, B, ..., F (or, sometimes, the lower case letters a .. f). The digit 'A' then has the decimal value 10, 'B' is 11 and so on. The transformation back into base ten works just like described in section \ref{sec:BinaryNumbers}. To make sure a number is given in hexadecimal, it often carries the prefix \texttt{0x}. More often than not, the exact value of addresses is of little importance; if you ever see hexadecimal numbers, simply know that they are just another way of writing normal decimal numbers.

As humans, we prefer symbols over addresses. \emph{The value of cell 238} doesn't tell a lot about its content or what it means. In contrast, \emph{the value of \texttt{nameOfPlayer1}} clearly signifies some textual information. We call \texttt{nameOfPlayer1} a \emph{symbol}. It is a stand-in for the address of the cell(s), in which the name of player 1 can be found. The compiler \emph{binds} the symbols we as humans choose to the addresses in memory.

\begin{defbox}[Memory Image]
\begin{center}
\begin{tikzpicture}
  [
    cell/.style={text width=8mm,
      text height=4mm, draw=black, inner sep=1mm},
    ld/.style={draw=blue,shorten >=2pt,->}
  ]
  \node (c1) at (0,0) [cell] {\ttfamily 99};
  \node (c2) at (1,0) [cell] {\ttfamily 1};
  \node (c3) at (2,0) [cell] {\ttfamily 255};
  \node (c4) at (3,0) [cell] {\ttfamily 0};
  \node (c5) at (4,0) [cell] {\ttfamily 80};
  \node (c6) at (5,0) [cell] {\ttfamily ...};

  \node (labelMem) at (8,  1) {Symbols in our Code};
  \node (labelMem) at (8,  0) {Values in memory};
  \node (labelMem) at (8, -1) {Addresses};

  \node (a1) [below=2mm of c1]            {\tiny 0x27ff};
  \node (a2) [below=2mm of c2, color=red] {\tiny 0x2800};
  \node (a3) [below=2mm of c3]            {\tiny 0x2801};
  \node (a4) [below=2mm of c4]            {\tiny 0x2802};
  \node (a5) [below=2mm of c5]            {\tiny 0x2803};
  \node (a6) [below=2mm of c6]            {\tiny 0x2804};

  \node (ptr) [below=8mm of c1] {\scriptsize Address of \texttt{x}};
  \node (vc2) [above=6mm of c1] {\scriptsize Variable \texttt{x}};
  \node (vc0) [above=2mm of c1] {\scriptsize Variable \texttt{y}};

  \draw [ld] (ptr.east) .. controls +(0.3,0) .. (a2.south);
  \draw [ld] (vc0.east) .. controls +(0.4,0) .. (c2.north);
  \draw [ld] (vc2.east) .. controls +(2.4,0) .. (c4.north);
\end{tikzpicture}
\end{center}
\end{defbox}

%TODO
% essentially keep structure of chapter 2, ...
% but introduce mem structure and pointers here already (take from chapter 3)
% C1: add "compile chameleon" to the exos

\subsection{Variablen}
Eine Variable hat einen Namen, einen \emph{Datentyp} und einen Wert.

Der Name einer Variablen ist eine beliebige Kombination aus \emph{alphanumerischen Zeichen}\footnote{Die Buchstaben a-z, A-Z, die Ziffern 0-9 sowie der Unterstrich (\_), jedoch keine Umlaute, Sonderzeichen oder das scharfe ß.}. Er muss mit einem Buchstaben beginnen, darf keine Leerzeichen enthalten und kann bis zu 40 Zeichen lang sein. Auch hier wird zwischen Groß- und Kleinschreibung unterschieden. Variablennamen müssen \emph{eindeutig} sein, \ie es darf keine zwei Variablen mit demselben Namen geben. Auch Funktionsnamen gelten als \enquote{verbraucht}; wir können keine Variable \texttt{main} nennen.

Anfänger neigen dazu kurze, nichtssagende Variablennamen zu vergeben. Um Programme gut les- und wartbar zu machen (und um somit Fehler zu vermeiden) sollte der Variablenname bereits ihre Funktion im Code angeben. Statt \texttt{a, b, c} also lieber \texttt{elementCount, averageNote, currentIndex}.

Unter \emph{Datentyp} versteht man die Art der Information, die in der Variablen gespeichert wird, \eg Ganzzahl, Kommazahl oder Text. Dies ist notwendig, da der Computer intern nur Einsen und Nullen speichern kann. Ohne Kontext ist es also nicht möglich, zu erkennen, wie eine Bit-Folge zu interpretieren ist. Dieser Kontext ist durch Zuweisung eines Datentyps gegeben.

Der Wert einer Variablen kann schließlich alles sein, was ihrem Datentypen entspricht; eine Variable kann \eg eine einzelne Zahl oder ein ganzes Bild speichern.

\subsection{Declaring Variables} \label{sec:DeclareVars}
To use variables in our code, we first have to \emph{declare} them. That means we have to tell the compiler that we want a variable and what kind of variable this should be. We do that with a statement of the form:
\begin{codebox}[Syntax: dDeclaring variables]
\texttt{dataType symbolName;}
\end{codebox}


Mehrere Variablen können in einem Schritt deklariert werden, indem man sie durch Kommata voneinander abtrennt.

Für den Moment wollen wir uns auf die Datentypen \mintinline{c}{int} (Ganzzahlen) und \mintinline{c}{double} (Kommazahlen) beschränken. Mit diesen können wir also folgende Variablen deklarieren:
\begin{codebox}[Beispiel: Deklaration von mehreren Variablen]
\begin{minted}[linenos]{c}
int main () {
   int    Ganzzahl;
   double Kommazahl, andereZahl;
}
\end{minted}
\end{codebox}
Variablen werden also \emph{innerhalb von Funktionen} deklariert\footnote{In Abschnitt \ref{sec:globals} werden wir hierzu mehr hören; für den Moment beschränken wir uns auf den Fall, dass alle Variablen am Anfang der Funktion \texttt{main} deklariert werden.}.

Ihnen mag die Ähnlichkeit zwischen der Deklaration von Ganzzahl-Variablen und der Funktion \texttt{main} aufgefallen sein. Dies ist kein Zufall. Tatsächlich ist formell die Funktion \texttt{main} die Vorschrift einen \mintinline{c}{int}-Wert zu berechnen. Das Symbol \texttt{main} ist eine Variable, die auf diese Vorschrift verweist. Hierzu mehr in Kapitel \ref{chp:OS-Link}.

\subsection{Wertezuweisung} \label{sec:valueAssignment}
Werte können über den \emph{Operator} = zugewiesen werden:
\begin{codebox}[Beispiel: Wertzuweisung]
\begin{minted}[linenos]{c}
int main () {
  int    Ganzzahl;
  double Kommazahl, andereZahl;

  Ganzzahl   = 7;
  Ganzzahl   = 8;
\end{minted}
\end{codebox}

\begin{codebox}[]
\begin{minted}[linenos, firstnumber=last]{c}
   Kommazahl  = 4.3;
   andereZahl = 5.0;
}
\end{minted}
\end{codebox}

In Zeile 5 wird also der Variablen \texttt{Ganzzahl} der Wert 7 zugewiesen. In der folgenden Zeile 6 überschreiben wir diesen Wert mit dem Wert 8.

In den Zeilen 7 und 8 erhalten die beiden \mintinline{c}{double}-Variablen ebenfalls Werte. Wir bemerken, dass hier ein \emph{Dezimalpunkt} gesetzt wird, also kein Komma.

Deklaration und Wertzuweisung können auch in einem einzigen Schritt geschehen:
\begin{codebox}[Beispiel: Kombinierte Deklaration und Wertzuweisung]
\begin{minted}[linenos]{c}
int main () {
   int    Ganzzahl   = 8;
   double Kommazahl  = 4.3,
          andereZahl = 5.0;
}
\end{minted}
\end{codebox}

\begin{hintbox}[Nicht-Initialisierte Variablen]
Wenn eine Variable deklariert wird, sorgt der Compiler lediglich dafür, dass genug Platz im Speicher reserviert wird, um diesen Wert zu erfassen. Im Allgemeinen wird aber kein bestimmter \enquote{Startwert} festgelegt. Die neue Variable startet mit dem Wert, den das Bitmuster an der Stelle festlegt, an der Platz für die Variable reserviert wurde. Dies ist ein zufälliger Wert! Programme, die mit einem solchen Zufallswert weiterarbeiten haben generell \emph{undefiniertes} Verhalten und können abstürzen oder sogar Schaden anrichten. Es ist daher also ratsam, allen Variablen bei der Deklaration gleich einen sinnvollen oder sicheren Startwert zuzuweisen.
\end{hintbox}

\begin{hintbox}[Fließkommazahlen ohne Nachkommastelle]
In der Mathematik gilt $5 = 5.0$. Ein Computer arbeitet zwar nach mathematischen Prinzipien; streng genommen gilt diese Gleichheit für den Prozessor aber nicht. Wie wir in Abschnitt \ref{sec:DataRepresentation} sehen werden kann dieselbe (mathematische) Zahl auf verschiedene Weisen binär dargestellt werden. Insbesondere zwischen Ganzzahlen und Kommazahlen bestehen große Unterschiede in der Binär-Darstellung. Der gcc kann zwar Umwandlungen automatisch durchführen, und würde auch
\begin{codebox}[Beispiel: Implizite Typumwandlung von Ganzzahl zu Fließkommazahl]
\begin{minted}[linenos]{c}
int main () {
   double Kommazahl = 5;
}
\end{minted}
\end{codebox}
\enquote{korrekt} umsetzen. Hier wird versucht, die \emph{Ganzzahl} \texttt{5} in der \mintinline{c}{double}-Variablen \texttt{Kommazahl} zu speichern. Da dies direkt nicht möglich ist, wandelt der gcc diese Ganzzahl zuerst in die Fließkommazahl \texttt{5.0} um und speichert dann diese. Eine solche \emph{implizite Typumwandlung} kann aber unbeabsichtigte Nebeneffekte haben und sollte daher vermieden werden. Als Beispiel für solche Nebeneffekte möchte ich Ihnen das folgende Beispiel geben:
\end{hintbox}
%
\begin{hintbox}[]
\begin{codebox}[Beispiel: Implizite Typumwandlung von Fließkommazahl zu Ganzzahl]
\begin{minted}[linenos]{c}
int main () {
   int Ganzzahl = 5.5;
}
\end{minted}
\end{codebox}
Analog zum Beispiel oben wird nun implizit die Zahl \texttt{5.5} zu einer Ganzzahl konvertiert. Natürlich ist Ihnen bewusst, dass es keine ganze Zahl gibt, die gleich $5.5$ ist. Es wird also gerundet. Im Gegensatz zur menschlichen Intuition rundet der gcc hier aber \emph{ab}. Außerdem können die \enquote{fehlenden} $0.5$ das Verhalten des Programms in der Laufzeit gegenüber der Erwartung verändern.

Damit der Datentyp (und damit das Verhalten des Programms) immer ersichtlich ist, sollten daher \enquote{ganzzahlige Fließkommawerte} als Kommawert (\eg \texttt{5.0}) gesetzt werden.
\end{hintbox}

\subsection{Rechnen mit Variablen -- Operatoren}\label{sec:OperatorsArithmetic}
Mit Variablen kann auch gerechnet werden. Hierzu stehen 6 \enquote{Grundrechenarten} zur Verfügung (siehe Tabelle \ref{tab:OperatorsArithmetic}). Mehrere solche Rechenoperationen können hintereinander geschaltet werden. Es gilt wie üblich Punkt-vor-Strich; Klammern können gesetzt werden um eine andere Reihenfolge festzulegen.
\begin{table}[h!]
	\newcolumntype{C}{>{\ttfamily\centering\arraybackslash} p{.2\linewidth}}
\begin{center}
\begin{tabularx}
	{.95\linewidth}
	{cC|cC}
\toprule[1pt]

	Operation   & \normalfont Zeichen  &  Operation                              & \normalfont Zeichen
\tabcrlf
	Addition    & +                    &  Multiplikation                         & * \\
	Subtraktion & -                    &  Division                               & / \\
	Negation    & -                    &  Rest der Division (\enquote{Modulo})   & \%\\

\bottomrule[1pt]
\end{tabularx}
\end{center}
\caption{Rechenoperatoren in C}\label{tab:OperatorsArithmetic}
\end{table}

\begin{codebox}[Beispiel: Rechnen mit Variablen]
\begin{minted}[linenos]{c}
int main () {
   int    Ganzzahl;
   double Kommazahl, andereZahl;

   Ganzzahl   = 7;
   Ganzzahl   = Ganzzahl + 8;

   Kommazahl  = 4.3;
   andereZahl = 3.0 * (-Kommazahl + 7.9) / 2.0;
}
\end{minted}
\end{codebox}

Wichtig ist, dass Variablen bereits deklariert sind \emph{bevor} auf sie verwiesen wird, \ie bevor sie in anderen Ausdrücken vorkommen. Folgendes Beispiel funktioniert daher nicht:
\begin{codebox}[Beispiel: Fehlerhafter Code{,} Variablen zu spät deklariert]
\begin{minted}[linenos]{c}
int main () {
  Ganzzahl = 7;
  int Ganzzahl;
}
\end{minted}
\end{codebox}
und erzeugt die folgende Fehlermeldung:
\begin{cmdbox}[Fehlermeldungen des gcc]
\begin{minted}{text}
myProgram.c: In function ‘main’:
myProgram.c:2:4: error: ‘Ganzzahl’ undeclared (first use in this
function)
    Ganzzahl = 7;
    ^~~~~~~~
myProgram.c:2:4: note: each undeclared identifier is reported only once
for each function it appears in
\end{minted}
\end{cmdbox}
Jede Fehlermeldung beginnt mit einem Verweis auf die fehlerhafte Datei und die Funktion darin, in der das Problem aufgetreten ist (\texttt{myProgram.c: In function ‘main’:}). In der folgenden Zeile der Fehlerausgabe wird genauer spezifiziert in welcher Zeile, und an welcher Stelle der Fehler auftrat (\texttt{myProgram.c:2:4}, also  Zeile 2, Spalte 4) und welcher Fehler gefunden wurde (\texttt{‘Ganzzahl’ undeclared}). Schließlich gibt der Compiler noch die relevante Stelle aus und markiert diese erfreulicherweise.

\begin{hintbox}[Zur Division mit Ganzzahl- und Fließkomma-Werten]
Der Datentyp einer Division hängt vom Datentyp von Divisor und Dividend ab. Ist einer der beiden ein Fließkommawert, so wird auch das Ergebnis als Fließkommawert berechnet. Sind beide Werte Ganzzahlen, wird auch das Ergebnis als Ganzzahl berechnet und gegebenenfalls abgerundet. Betrachten wir folgendes Beispiel:
\begin{codebox}[Beispiel: Division von Ganzzahlen und Fließkommazahlen]
\begin{minted}[linenos]{c}
double x = 3   / 2  ,
       y = 3.0 / 2.0,
       z = 3.0 / 2  ,
       w = 3   / 2.0;
\end{minted}
\end{codebox}
Die Variable \texttt{x} geht aus der Division zweier Ganzzahlen hervor, und wird damit zu \texttt{1.0} abgerundet. Bei der Berechnung von \texttt{y} hingegen gehen Fließkommazahlen in die Division ein; somit wird das Ergebnis nicht gerundet und als 1.5 gespeichert. Dasselbe gilt für \texttt{z} und \texttt{w}, da zumindest \emph{eine} der an der Division beteiligten Zahlen eine Fließkommazahl war.
\end{hintbox}

\subsection{Shorthands}
Häufig soll sich der Wert einer Variablen \emph{inkrementiert} (\enquote{hochgezählt}) werden. Dies kann codiert werden als:
\begin{codebox}[Beispiel: Inkrementieren einer Variablen i]
\begin{minted}[linenos]{c}
int main () {
   int i = 5;
   i = i + 1;  // i hat jetzt den Wert 6
}
\end{minted}
\end{codebox}

Daneben gibt es auch den \emph{Shorthand} (Kurzform) \texttt{variable++}:
\begin{codebox}[Beispiel: Inkrementieren einer Variablen i mit Shorthand ++]
\begin{minted}[linenos]{c}
int main () {
   int i = 5;
   i++;       // i hat jetzt den Wert 6
}
\end{minted}
\end{codebox}

In beiden Beispielen wird der Wert von \texttt{i} in Zeile 3 um 1 erhöht.

Analog dazu existiert der \emph{Dekrement}-Operator \texttt{variable-{}-}, der den Wert von \texttt{variable} um 1 verringert.

Weiter existieren der Operator-Shorthand \texttt{variable += expression}. Zu \texttt{variable} wird der Wert von \texttt{expression} hinzugezählt und anschließend in \texttt{variable} gespeichert. \texttt{expression} darf dabei ein beliebig komplexer Ausdruck sein:
\begin{codebox}[Beispiel: Inkrementieren einer Variablen i mit Shorthand \texttt{+=}]
\begin{minted}[linenos]{c}
int main () {
   int i = 5,
       j = 2;
   i += 3 * j + 1;     // i = i + 3 * j + 1
}
\end{minted}
\end{codebox}
Hier wird zunächst der Ausdruck \texttt{3 * j + 1} ausgewertet zu \texttt{7}; als nächster Schritt wird die Summe \texttt{i + 7} berechnet, und diese dann in \texttt{i} gespeichert. Die Variable \texttt{i} hat also nach Zeile 4 den Wert 12.

Analog existieren auch die Shorthands \texttt{-=}, \texttt{*=}, \texttt{/=} und \texttt{\%=}.

\begin{hintbox}[Inkrement und Dekrement: Prefix und Postfix]
Die Operatoren \texttt{++} und \texttt{-{}-} können sowohl \emph{vor} als auch \texttt{hinter} einen Ausdruck gesetzt werden (wir sprechen von \emph{Prefix} und \emph{Postfix}). Für sich alleine ist in beiden Fällen der Effekt derselbe -- der Wert des Ausdrucks wird um 1 erhöht bzw. verringert. Kombiniert mit anderen Operationen ergeben sich aber Unterschiede. Betrachten wir folgendes Beispiel:

\begin{codebox}[Beispiel: Unterschiede bei Prefix- und Postfix-Inkrement]
\begin{minted}[linenos]{c}
int main () {
   int i, j;

   // Postfix-Form
   i = 5;
   j = i++;		// i = 6, j = 5.

   // Prefix-Form
   i = 5;
   j = ++i;		// i = 6, j = 6.
}
\end{minted}
\end{codebox}

In der Postfix-Form (Zeile 6) wird zuerst der Wert von \texttt{i} gelesen und der Variablen \texttt{j} zugewiesen; danach wird \texttt{i} inkrementiert.

Bei der Prefix-Form (Zeile 10) dagegen wird zuerst \texttt{i} inkrementiert und danach der Wert von \texttt{i} in \texttt{j} gespeichert.
\end{hintbox}
%
\begin{hintbox}[]
Um den Code leicht les- und wartbar zu halten, empfehle ich, alle gedanklichen Schritte in abgeschlossene Befehle zu setzen. Äquivalent zum obigen Code ist der folgende, intuitiv leichter verständliche Code:

\begin{codebox}[Beispiel: Äquivalente Codes]
\begin{minted}[linenos]{c}
int main () {
   int i, j;

   // zur Postfix-Form
   i = 5;
   j = i;
   i++;		// i = 6, j = 5.

   // zur Prefix-Form
   i = 5;
   i++;
   j = i;		// i = 6, j = 6.
}
\end{minted}
\end{codebox}
\end{hintbox}

\section{Formatierte Ausgabe}\label{sec:formattedOutput}
Wir wollen nun auch die berechneten Ergebnisse in Erfahrung bringen. Dazu können wir den bereits bekannten Befehl \texttt{printf} benutzen. Betrachten wir folgendes Beispiel:

\begin{codebox}[Beispiel: Ausgabe einer Berechnung mit \texttt{printf}]
\begin{minted}[linenos]{c}
#include <stdio.h>

int main () {
   double result = 5.0 / 7.0;

   printf("5.0 / 7.0 = %lf\n", result);
}
\end{minted}
\end{codebox}

Die Ausgabe lautet:
\begin{cmdbox}[Ausgabe einer Berechnung mit \texttt{printf}]
5.0 / 7.0 = 0.714286
\end{cmdbox}

Wie oben besprochen wird zuerst in Zeile 4 das Ergebnis der Division zweier Fließkommazahlen berechnet und in der Variablen \texttt{result} gespeichert. Zur Ausgabe auf dem Bildschirm übergeben wir dem Befehl \texttt{printf} jetzt \emph{zwei} Argumente:

Das erste Argument wird \emph{Format-String} genannt und ist in unserem Beispiel der Anteil:
\begin{center}
\mintinline{c}{"5.0 / 7.0 = %lf\n"}.
\end{center}

Ein Format-String ist eine Zeichenkette die \emph{Platzhalter} enthalten kann. Diese beginnen mit dem Zeichen \texttt{\%} gefolgt von einem oder mehreren Zeichen, die beschreiben, wofür Platz freigehalten werden soll. In diesem Fall steht der Platzhalter \texttt{\%lf}, was für eine \mintinline{c}{double}-Zahl steht. Diese Zahl wird hinter dem Format-String genannt und von diesem durch ein Komma abgetrennt. Weitere Platzhalter-Symbole sind in Tabelle \ref{tab:FormatOutNum} aufgelistet.

In einem Format-String können beliebig viele Platzhalter stehen; für jeden Platzhalter muss eine entsprechende Einsetzung hinter dem Formatstring selbst genannt werden.
\begin{codebox}[Beispiel: Ausgabe mehrerer Werte mit \texttt{printf}]
\begin{minted}[linenos]{c}
#include <stdio.h>

int main () {
   double f = 3.14;
   int    g = 42;

   printf(
      "Ganzzahl i = %d, Fliesskommazahl d = %lf, weitere Zahl = %d",
                    g,                      f,                  99);
}
\end{minted}
\end{codebox}

Werden hinter dem Formatstring zu viele oder zu wenige Werte angegeben, so meldet der Compiler eine Warnung:
\begin{cmdbox}[Warnung bei zu wenigen Argumenten für den Formatstring]
\begin{minted}{text}
myProgram.c: In function ‘main’:
myProgram.c:4:12: warning: format ‘%i’ expects a matching ‘int’ argument
[-Wformat=]
   printf("%i\n");
           ~^
\end{minted}
\end{cmdbox}
Das Programm wird jedoch \enquote{erfolgreich} kompiliert und ausgeführt. Anstelle des Platzhalters wird ein zufälliger Wert vom Speicher eingesetzt\footnote{Bugs dieser Art sind leider häufig. Der im Jahr 2014 entdeckte \emph{Heartbleed-Bug} in der OpenSSL-Library geht auf einen ähnlichen Mechanismus zurück. Hacker konnten \enquote{fehlerhafte Anfragen} an einen Server schicken und so den Speicherinhalt des Servers auslesen -- und damit \eg Zugangsdaten der User auslesen. Compiler-Warnungen sollten also sehr ernst genommen werden.}.

Ähnliches stellen wir fest, wenn wir das folgende Beispiel betrachten:

\begin{codebox}[Beispiel: Ausgabe mehrerer Werte mit \texttt{printf}]
\begin{minted}[linenos]{c}
#include <stdio.h>

int main () {
   printf("%d\n", 1.5);
}
\end{minted}
\end{codebox}
Der Formatstring \texttt{\%d} beschreibt die Ausgabe einer \emph{Ganzzahl}; als Wert für diesen Platzhalter wird aber eine Fließkommazahl angeboten. Die Compiler-Warnung:

\begin{cmdbox}[Warnung bei falscher Zuordnung Platzhalter/Datensatz]
\begin{minted}{text}
myProgram.c: In function ‘main’:
myProgram.c:4:12: warning: format ‘%d’ expects argument of type ‘int’, but
argument 2 has type ‘double’ [-Wformat=]
   printf("%d\n", 1.5);
           ~^
           %f
\end{minted}
\end{cmdbox}
weist darauf hin, dass die Ausgabe mit \texttt{\%d} für Fließkommawerte ungeeignet ist (und weist auf die Möglichkeit hin, mit \texttt{\%f} auszugeben). Trotz dieses Fehlers kann das Programm umgesetzt und erfolgreich ausgeführt werden. Angezeigt wird aber nicht \texttt{1.5}, sondern \texttt{-2102427368}. Diese Zahl ergibt sich, wenn das Bitmuster der Zahl \texttt{1.5} als Ganzzahl interpretiert wird\footnote{Genauer: Die ersten 32 bit des Bitmusters; \texttt{1.5} ist eine double-Zahl und damit 64 bit breit. In jedem Fall ist diese Ausgabe \enquote{unsinnig}.}.

Wir hatten zuvor schon \texttt{\textbackslash n} als Umschreibung des Zeilenumbruchs kennengelernt. Diese Umschreibung bestimmter Zeichen nennen wir \emph{Escape-Sequence}. Solche Escape-Sequenzen werden benutzt um Zeichen einzufügen, die Teil der C-Syntax sind und sonst nur schwer auf dem Bildschirm ausgegeben werden können. In diesem Kurs werden wir \texttt{\textbackslash n} (Zeilenumbruch), \texttt{\textbackslash \textbackslash} (Backslash) und \texttt{\textbackslash ''} (Anführungszeichen) häufiger sehen.

In den Tabellen \ref{tab:FormatOutNum} und \ref{tab:FormatOutSpc} im Anhang sind die wichtigsten Codes für Platzhalter und Escape-Sequenzen aufgelistet.

Formatstrings legen nicht nur den \emph{Datentyp} der Ausgabe fest. Zusätzlich kann eine Information hinterlegt werden, wie viel Platz auf dem Bildschirm benutzt werden soll. Auf diese Weise können tabellarische Ausgaben stattfinden. Zu diesem Zweck setzt man zwischen das \texttt{\%} und dem Platzhalter-Symbol (\eg \texttt{d} für Ganzzahl) eine Zahl:

\begin{codebox}[Beispiel: Format-String mit Platzangabe (1)]
\begin{minted}[linenos]{c}
#include <stdio.h>

int main () {
   printf("Text %3d\n", 1);
   printf("Text %3d\n", 10);
   printf("Text %3d\n", 100);
   printf("Text %3d\n", 1000);
}
\end{minted}
\end{codebox}

\begin{cmdbox}[Ausgabebeispiel: Format-String mit Platzangabe (1)]
\begin{minted}{text}
Text   1
Text  10
Text 100
Text 1000
\end{minted}
\end{cmdbox}

Sie sehen also dass die ersten drei Zeilen rechtsbündig ausgegeben. Dies liegt daran, dass mit \texttt{\%3d} drei Textzeichen für einen Ganzzahl-Wert vorgesehen werden. Die vierte Zeile enthält eine Zahl, die länger ist als die vorgesehenen drei Zeichen. Daher erfolgt die Ausgabe, als wäre keine Zahl eingefügt.

Es gibt auch die Möglichkeit, Platz linksbündig zu reservieren. In diesem Fall gibt man negative Zahlen an:


\begin{codebox}[Beispiel: Format-String mit Platzangabe (2)]
\begin{minted}[linenos]{c}
#include <stdio.h>

int main () {
   printf("Text %-3d Text\n", 1);
   printf("Text %-3d Text\n", 10);
   printf("Text %-3d Text\n", 100);
   printf("Text %-3d Text\n", 1000);
}
\end{minted}
\end{codebox}

\begin{cmdbox}[Ausgabebeispiel: Format-String mit Platzangabe (2)]
\begin{minted}{text}
Text 1   Text
Text 10  Text
Text 100 Text
Text 1000 Text
\end{minted}
\end{cmdbox}

Bei Fließkommazahlen können wir zusätzlich zum Gesamtplatz für die Ausgabe auch angeben, wie viele Stellen hinter dem Komma angegeben werden sollen. Die Ausgabe wird dabei automatisch gerundet, wenn weniger Nachkommastellen angegeben werden als die auszugebende Zahl:

\begin{codebox}[Beispiel: Format-String mit Platzangabe (3)]
\begin{minted}[linenos]{c}
#include <stdio.h>

int main () {
   printf("Text %5.3lf Text\n", 1.5);
   printf("Text %5.3lf Text\n", 1.555);
   printf("Text %5.3lf Text\n", 1.5555);
   printf("Text %5.3lf Text\n", 15.0);
}
\end{minted}
\end{codebox}

\begin{cmdbox}[Ausgabebeispiel: Format-String mit Platzangabe (3)]
\begin{minted}{text}
Text 1.500 Text
Text 1.555 Text
Text 1.556 Text
Text 15.000 Text
\end{minted}
\end{cmdbox}

\section{Kommentare}
Im Optimalfall ist Code selbsterklärend. \enquote{Sprechende} Variablennamen und gute Struktur erlauben, viele Zusammenhänge intuitiv zu erfassen. Dennoch ist es oft notwendig (und dann auch guter Stil), Programme mit Kommentaren zu versehen.

Kommentare können auf zwei Arten gesetzt werden:
\begin{itemize}
\item Mehrzeilige Kommentare beginnen mit \texttt{/*} und enden mit \texttt{*/}. Achtung: Kommentare können nicht \enquote{verschachtelt} werden!
\item Einzeilige Kommentare werden durch zwei \emph{forward slashes} \texttt{//} eingeleitet. Sie enden mit dem nächsten Zeilenumbruch\footnote{Diese Form der Kommentare ist erst seit dem Standard C99 aus dem Jahr 1999 zulässig; davor existierte nur die \enquote{Sternchen-Form}.}.
\end{itemize}

Ich empfehle, auch bei Kommentaren keine länderspezifischen Zeichen zu verwenden, also z.B. keine Umlaute (\emph{äöü}). Am besten verwenden Sie nur Zeichen nach ASCII Tabelle Abb.\,\ref{fig:ASCII}.

\begin{codebox}[Beispiel: Kommentare]
\begin{minted}[linenos]{c}
int main () {
   int i = 0;
   i++;  // Inkrement-Shorthand
   // Das ist noch ein einzeiliger Kommentar.

   /* Kommentar der sich über mehrere Zeilen erstreckt.
    * Oft setzt man weitere Sternchen an den Zeilenanfang, dies
    * dient aber nur optischen Zwecken und kann
      nach belieben ausgelassen werden.
    */

   /* "Stern-Form" auch für einzeilige Kommentare */

   /* Ein weiterer Kommentar
    *  /* FEHLER -- keine verschachtelten Kommentare */
    */
}
\end{minted}
\end{codebox}

\section{Hello World in C++}
\begin{plusbox}
Die hier gegebenen Erklärungen und Optionen gelten so auch für C++. Das bedeutet, dass der oben gezeigte Code auch in C++ so kompiliert werden kann und dasselbe Ergebnis erzeugt. Jedoch steht hier für die Textausgabe auch das Konzept \emph{Streams} zur Verfügung. Betrachten wir den folgenden Code:

\begin{codebox}[Beispiel: Hello World in C++]
\begin{minted}[linenos]{c++}
#include <iostream>

int main() {
   std::cout << "Hello world!" << std::endl;
}
\end{minted}
\end{codebox}

Wie schon zuvor wird mit \mintinline{c++}{#include} ein Header geladen (Achtung: Hier keine Erweiterung \texttt{.h}!). In diesem Fall ist es der Header, der Input- und Output-Streams definiert. Ein \emph{Stream} ist ein Programm-Interface, das Daten in verschiedenen Formaten annehmen und verarbeiten kann.

Die Deklaration des Hauptprogramms \texttt{main} verläuft völlig analog zur Form in C.

\texttt{std::cout} ist der \enquote{Ausgabe-Stream}. Mit dem Operator \texttt{<{}<} können Datenobjekte an den Stream übergeben werden. Mehrere Datenobjekte können \enquote{hintereinander} in den Stream geschickt werden. So wird hier nach dem Text \texttt{Hello World!} ein Zeilenumbruch \texttt{std::endl} an den Stream übergeben.

Platzhalter sind bei der Arbeit mit \texttt{std::cout} nicht notwendig.
\end{plusbox}

\begin{plusbox}[]
Um weitere Details für die Ausgabe festzulegen siehe:
\begin{itemize}
\item \url{https://en.cppreference.com/w/cpp/io/manip/setw} -- Gesamtbreite für die Ausgabe einer Zahl
\item \url{https://en.cppreference.com/w/cpp/io/manip/setprecision} -- Anzahl der Nachkommastellen
\item \url{https://en.cppreference.com/w/cpp/io/manip/setiosflags} -- Ausgabeformat (exponentialschreibweise, hexadezimal, ...)
\end{itemize}
Siehe außerdem \url{https://en.cppreference.com/w/cpp/io/cout}
\end{plusbox}
