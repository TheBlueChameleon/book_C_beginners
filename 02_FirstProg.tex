\chapter{Writing C Code}
\epigraph{The secret to getting ahead is getting started.}{Mark Twain}

In this chapter, we'll discuss the principal structure of a C language program by analyzing a so called \emph{Hello World} -- a program which has the sole purpose of putting the text \emph{Hello World} on screen\footnote{The expressin \emph{Hello World} is also used in context of other languages. This \enquote{most simple} possible program is commonly used as a sort of reference, a way to compare programming languages -- or to simply fool around. People have made s sport out of writing programs that essentially are Hello World variations but don't look like anything made for human eyes at all. See, for example, \url{https://codegolf.stackexchange.com/questions/22533/weirdest-obfuscated-hello-world}}.

Once we've familiarized ourselves with this primary structure, we'll start writing actually useful code by making simple calculations in C.

\section{A First Program: Hello World}

The following lines show code that prints the text \texttt{Hello World} on screen:
\begin{codebox}[helloworld.c]
\begin{minted}[linenos]{c}
#include <stdio.h>

int main() {
   printf("Hello World!\n");
}
\end{minted}
\captionof{code}{A Hello World} \label{code:helloworld}
\end{codebox}

To compile and run it, remember the compiler invocation:

\begin{cmdbox}[Compiling and Executing the Hello World]
\$ gcc -std=c17 -Wall -Wextra -Wpedantic helloWorld.c -o helloWorld \\
\$ ./helloWorld \\
Hello World!
\end{cmdbox}

Now let's go through code \ref{code:helloworld} line by line.

We've already discussed that recurrent tasks can be externalized and put into a library. Input and Output fram and to the console is such a recurrent task. The routines concerned with these tasks are implemented in the \emph{C standard library}, which is automatically included in the linkage process -- no need to use a compiler option. However, these pre-compiled routines are, well, compiled, \ie translated into machine language. In this process, some information is lost that would allow for direct integration in our code. To provide this lost information, a so called \emph{Header File} needs to be \emph{included} into our code. For the time being, don't concern yourself with what this means in detail too much; simply think of a header file as a sort of glue that helps combining your C-Code to machine language code somewhere else.

After these explanations, you can probably guess that in line 1 we include this header file. The command \texttt{\#include} loads a file from the hard disk, and copy-pastes it into the C-code file. The file to load is specified in <angle brackets>\footnote{In chapter \ref{chp:Preprocessor}, we'll see that there are other options, too -- but let's not get ahead of ourselves.} This also constitutes a preprocessor operation: line 1 is replaced by the content of the file \texttt{stdio.h} by the preprocessor. All \emph{preprocessor directives} begin with a pound symbol (\texttt{\#}).

Programs are organized into subroutines or \emph{functions}, \ie smaller sections of code that form the building blocks of the whole. Line 3 \emph{declares} such a function. It has the name \texttt{main}. The \emph{body} of the function (\ie its \enquote{content}) is enclosed by \{curly braces\}. All code but a very few needs to be placed within functions, so we'll see this pattern in every program we'll see throughout our C coding days. Functions can have virtually any name, but the name \texttt{main} is special: it marks the, well, \emph{main} function, \ie the starting point for our program. When there are multiple functions, all of them but the \texttt{main} function are \enquote{ignored} until you as a programmer explicitly \emph{call} them -- but we'll hear more about that in chapter \ref{chp:funcs}. For now, accept simply that the line \inC{int main ()} marks the starting point for the \enquote{proper} code, which you will find within \{braces\}.

The only line left to understand is line 4. It contains a \emph{statement}, \ie a command that can be executed, and this statement is a \emph{call to the function \texttt{printf}}. That means, somewhere (in the precompiled standard library), there is another function with the name \texttt{printf}. The code there can be used to make output on screen (the \texttt{f} in the function's name stands for \emph{formatted} -- we'll see later, why that is). Of course, the routine needs to know \emph{what text} should be put on screen. To distinguish the \emph{function arguments} (\ie the information sent or \emph{passed} to the function) from the rest of the code, they are enclosed by (parens). In general, we can pass all kinds of information into functions. In the previous chapter, we've already heard that there is some ambiguity as to how to interpret \enquote{raw information}. In particular, we've seen that text is treated differently from numbers. But also when we're dealing with text (like our C code), there can still be different ways to interpret them. \texttt{Hello Word!} could be the text we want to print on screen, or some C code that needs to be translated into machine language. Since the computer can't guess our intent (remember our mantra from the preface?), we need to be explicit: we mark the text we want to put on screen as \emph{literal text} by enclosing it in a pair of \texttt{"}quotation marks\texttt{"}.

While this makes it clear what is code and what is literal text, it also introduces some minor problems. Say, you want to have some text that also includes quotation marks. There needs to be a way to tell the compiler that a double quote in the code is not to be interpreted as end of text delimiter but as part of the text. This is done by \emph{escape sequences}: special substrings in a text that are not taken verbatim but interpreted in a different (simple) way. In C, escape sequences begin with a backslash (\textbackslash) followed by usually one chacracter. For example, the sequence \texttt{\textbackslash"} becomes such a quotation mark within the text.

In line four, we find the escape sequence \texttt{\textbackslash n}; this encodes a line break (\texttt{n} stands for \emph{new line}). Unfortunately, literal texts cannot span multiple lines, so for the time being, these escape sequences are the only way to make multiline output. See table \ref{tab:Escape} in the appendix for more escape sequences.

Note also that line four ends with a semicolon (;). This ends the statement. Each statement needs to end in a semicolon, otherwise, you'll find that gcc refuses to compile your code and outputs an error message:

\begin{cmdbox}[Error message when compiling \texttt{helloworld.c} without semicolon in line 4]
\begin{minted}{text}
helloworld.c: In function ‘main’:
helloworld.c:4:29: error: expected ‘;’ before ‘}’ token
    4 |     printf("Hello World!\n")
      |                             ^
      |                             ;
    5 | }
      | ~
\end{minted}
\end{cmdbox}

The numbers in the error message (\texttt{helloworld.c:4:29: ...}) tell you where in your code the error was found: line 4, column 29. Especially in larger projects, this is an essential piece of information for finding the location that caused the problem\footnote{Projects a single person manages easily grow to some ten or hundred thousand lines of code. Teams of coders can build projects with millions of lines of code.}.

This is because (unlike literal strings) statements can be spread over multiple lines. Similarly, there can be more than one statement in a single line. As programmers, this gives us some liberty in formatting the code such that it can be read more easily. Also, we can add or remove whitespaces (or tabs) pretty much anywhere, as long as we keep \enquote{words} together. The following code produces the same output as code \ref{code:helloworld}:
\begin{codebox}[whitespaces.c]
\begin{minted}[linenos]{c}
  #include     <stdio.h>
 int    main   (   )    {
      printf   (   
         "Hello World!\n"
)   ;      }
\end{minted}
\captionof{code}{A Hello World with added whitespaces} \label{code:optionalWhitespaces}
\end{codebox}

\begin{hintbox}[Rule of thumb: whitespaces]
The added whitespaces in code \ref{code:optionalWhitespaces} are meant to illustrate where it is allowed to add them; clearly, this example makes the code look more like a random image than an organized list of instructions. This is not to say that \emph{indenting} your code is a bad idea. Quite to the contrary, these whitespaces and extra line breaks can help you a lot in visualizing relationships between different parts of the code or filter relevant information. This becomes paramount once our codes reach some complexity.

As a rule of thumb, you should add a level of indentation for every opening brace (\{\}), paren () or bracket ([ ]) that is not closed in the same line. Furthermore, the second to last line of statements that take more than one line of code should also be indented.

It is up to you, how many whitespaces (or tabs) you add; only you should stay consistent throughout your code. Commonly, one indentation level means four white spaces, but also two or eight whitespaces are not unheard of. Many code editors can be configured to automatically indent to the next multiple of two, four or eight whitespaces on the press of the tabulator key.

In this book I'll do my best to stay consistent in my indentation style. You may take this as inspiration for your own code.
\end{hintbox}

The gcc reads the code \enquote{from top to bottom}. This means that the order of statements is of great importance. If, for example, you move the \texttt{\#include} statement from line 1 to the end of the code, you'd get an error message. The reason behind this is that the command \texttt{printf} is only declared in the fine \texttt{stdio.h}; before this declaration, the compiler \enquote{doesn't know} what \texttt{printf} even means.


\section{Comments}
Albeit sounding unintuitive, computer code (like our C code) is meant for \emph{human readers} (as opposed to \emph{machine language}, which is meant, well, for machines). While this means it is \emph{easier} to understand C code than it is to understand machine language, it does not imply that it is \emph{easy} to understand. While perfect code is self-explaining, we often want to leave some comments in natural language in our code, so that when we work on larger projects, we can more easily retrace our train of thought and remember which components have to be used in what way.

To that end, the C programming language allows to leave blocks of text that are simply ignored by the compiler. Adding such \emph{comments} to the code can be done in one of two ways: \vspace{-9pt}
\begin{itemize}
\setlength\itemsep{0pt}
\item By opening the comment with the characters \texttt{/*} and closing them with \texttt{*/}. Any text in between these two markers will be ignored, even if they are not in the same line.
	This form is often called a \emph{block quote}.
\item By introducing the comment with the characters \texttt{//}. The entire rest of the line will be ignored. You can think of the line break as the close-comment marker in this form.
\end{itemize}

Regard this example:
\begin{codebox}[helloworld.c]
\begin{minted}[linenos]{c}
/* Hello World */

/* Comments in the first lines of a program often describe the purpose of
 * the program or give contact and copyright information.
 *
 * In the "asterisk form", multiple lines can be combined into one single 
 * comment. The opening asterisk in each line after the opening of the
 * comment are optional and merely serve aesthetic purpose.
 */

// You can also use the double-slash variant, but each new line HAS TO 
// begin with a renewed double slash.

#include <stdio.h>                     // loads the definition of printf
// the double slash variant is often used to add brief remindes behind
// "proper code".

int main() {
   printf("Hello World!\n" /* the asterisk form can be interspersed
                              with regular code! */ );
   printf("/* This is NOT a comment! Quotation marks have higher priority */");
}
\end{minted}
\captionof{code}{Comments} \label{code:comments}
\end{codebox}

Note that you cannot \emph{nest} block quotes; all but the first instance of \texttt{/*} counts towards the comment and is thus ignored. Consequently, the first instance of \texttt{*/} ends the comment; subsequent instances of the same string of characters would be interpreted as proper code and constitute a syntax error.

\begin{warnbox}[nested block comments, leftupper=7mm]
\begin{minted}[linenos]{c}
int main () {
   /* This opens the block comment.
   /* All of this line is part of the comment that began one line above,
    * and it ends HERE. */
    * this will be taken for proper code and does not compile. */
}
\end{minted}
\end{warnbox}

\begin{plusbox}[History of comment syntax]
The single-line syntax for comments was only introduced in the standard from 1999. In legacy code, you'll only find the asterisk form. If, for some reason, you need to write code for an older compiler or with or a language standard older than C99, you may not use the double slash form.
\end{plusbox}

In the lessons to come we'll talk about best practices and coding conventions (rules that are not necessary from a technical point of view, but are obeyed nonetheless because they make life as a programmer easier). Abiding by these rules often saves us the hassle of typing and reading comments. After all, we don't want to put too much effort into text that is ignored by the computer. Still, it is a good idea to document hard to gras aspects right in the code.

What does or does not constitute a good comment is hard to define, and also depends on the experience of the author/reader. The comment in line 14 of listing \ref{code:comments} (\texttt{loads the definition of printf}) would be regarded unnecessary by every programmer except for us first day beginners; still, in this stage of our course it might be a nice reminder.

As a rule of thumb, comments should add to the code and not be redundant. A note like 
\vspace{-9pt}
\mint{c}{printf("hello world\n");     // prints "hello world"}
\vspace{-9pt}
is certainly superfluous. Better uses of comments will be discussed here once we know more coding concepts.

\begin{hintbox}[Style Guide]
\emph{Robert Cecil Martin} wrote a wonderful book titled \emph{Clean Code} (ISBN 978-0-13-235088-4). It is aimed at Java programmers with \emph{some} experience. Still, most of what he has to say applies to basically any programming language in existence. While right now, this book might not be very useful to you, definitely pick it up once you've finished this course. Among other things, it has one chapter dedicated to the question what is a good, bad or even harmful comment.

To double down on this recommendation, let me quote here the conversation in which I learned about this book:
\vspace{-6pt}
\begin{quote}
\hspace{-2pt}-- ... yeah, there's a chapter about this very problem in \emph{Clean Code} for a reason.

-- Clean Code?

-- You haven't heard about \emph{Clean Code yet}?

-- I'm afraid, no.

-- Okay. Go, get a copy. Now. Read it.

-- Okay, I'll give it a try.

-- This is your bible now.

-- Okay okay!
\end{quote}

... he was right.
\end{hintbox}

\section{Main Takeaways}
\begin{defbox}[You Should Remember...]
\textbf{The \texttt{\#include} directive(s)}
\begin{itemize}
\item loads definitions from other files
\item should make up the first line(s) of your code
\item loads \texttt{<stdio.h>} in almost all of our programs
\end{itemize}

\textbf{The \texttt{main} function}
\begin{itemize}
\item is enclosed by \texttt{\{}curly braces\texttt{\}}
\item contains the actually executed code
\item needs to be introduced by the text \texttt{int main} (case sensitive!)
\end{itemize}
\end{defbox}
%
\begin{defbox}[]
\textbf{The command \texttt{printf}}
\begin{itemize}
\item takes \emph{arguments}, which are mentioned in parentheses
\item needs strings to be in \texttt{"}double quotes\texttt{"}
\item interpretes the substring \texttt{\textbackslash n} as a line break
\item needs to be terminated by a semicolon (\texttt{;})
\end{itemize}

\textbf{Comments}
\begin{itemize}
\item come as block comments (\texttt{/* comment */})
\item or as single-line comemnts (\texttt{// comment}
\end{itemize}
\end{defbox}

\section{Exercises and Solutions}
Find the mistakes in the following listing:
\begin{codebox}[helloworld-faulty.c]
\begin{minted}[linenos]{c}
int Main {
   print('Hello World!\n' // \n means a line break )
   #inculde {stdio.c}
}
\end{minted}
\end{codebox}

\rule{\linewidth}{0.1mm}
\subsection*{Solution}
\begin{itemize}
\item Line 1:
	\begin{itemize}
	\item The primary function must be called \texttt{main}. C is case sensitive, \ie capitalization matters. \texttt{Main} is different from \texttt{main}.
	\item The empty parentheses after \texttt{main} are missing.
	\item In total, the entire line should be \texttt{int main () \{}
	\end{itemize}
\item Line 2:
	\begin{itemize}
	\item The command's name is \texttt{printf} (the \texttt{f} is missing)
	\item The text to be printed needs to be put in \emph{double} quotes (\texttt{"}) rather than single quotes (\texttt{'})
	\item The final semicolon (\texttt{;}) is missing
	\item The comment hides the closing paren. Either use a closed block-quote (\texttt{/* .. */}) or move the comment behind the command (\texttt{printf(...); // \textbackslash n means ...}
	\item In total, the entire line should be \texttt{printf("Hello World!\textbackslash n"); // \textbackslash n means  a line break}
	\item The command printf can only be used after it has been defined by \texttt{\#include}ing the header \texttt{stdio.h}.
	\end{itemize}
\item Line 3:
	\begin{itemize}
	\item The directive is called \texttt{\#include}
	\item The included file is given in \texttt{<}angle brackets\texttt{>}, not in \texttt{\{}curly brackets\texttt{\}}
	\item The included file should be \texttt{stdio.h}, not \texttt{stdio.c}
	\item In total, the line should be \texttt{\#include <stdio.h>}
	\item As mentioned before: this line should appear before the first use of \texttt{printf}, and best before the body of main; \texttt{\#include}-directives should be the 
		first lines of code, before the function \texttt{main} begins.
	\end{itemize}
\end{itemize}

Funny how much one can get wrong in so few lines of code, amirite?